\thispagestyle{empty}%
\begin{center}%
  \vspace*{\frontmattertitleskip}%
  \begin{doublespace}%
    {\Huge\textbf{\thetitle}}\\%
  \end{doublespace}%
  \vspace*{2cm}%
  \Large{{Dissertation \\
          zur Erlangung des Doktorgrades \\
          der Naturwissenschaften} }\\%
  \vspace*{2cm}%
  \Large{{vorgelegt beim Fachbereich Physik \\
         der Johann Wolfgang Goethe-Universit\"at \\
         in Frankfurt am Main} }\\%
  \vspace*{1cm}%
  \Large{{von {\theauthor} \\ aus Dnepropetrowsk} }\\
  \vspace*{1cm}%
  \Large{{Frankfurt 2017 \\ (D30)}}
\end{center}%

\newpage
\thispagestyle{empty}%
\vspace*{0.4\textheight}
\noindent
{vom Fachbereich Physik (13) der \\
Johann Wolfgang Goethe-Universit\"at als Dissertation angenommen.}\\
\vspace{5cm} \\
{Dekan:  Prof. Owe Philipsen\\
 Gutachter:  Prof. Hannah Petersen, Prof. Marcus Bleicher\\
 Datum der Disputation: }

%% Abstract
\newpage
\begin{abstract}[Zusammenfassung]%[\smaller \thetitle\\ \vspace*{1cm} \smaller {\theauthor}]
  %\thispagestyle{empty}
  Diese Arbeit basiert auf folgenden Publikationen:
  \begin{itemize}
    \item ``Systematic Investigation of Negative Cooper-Frye Contributions in Heavy Ion Collisions Using Coarse-grained Molecular Dynamics''~\cite{Oliinychenko:2014tqa}
    \item ``Cooper-Frye Negative Contributions in a Coarse-Grained Transport Approach''~\cite{Oliinychenko:2014ava}
    \item ``Deviations of the Energy-Momentum Tensor from Equilibrium in the Initial State for Hydrodynamics from Transport Approaches'' \cite{Oliinychenko:2015lva}
    \item ``Influence of kinematic cuts on the net charge distribution''~ \cite{Petersen:2015pcy}
    \item ``Particle production and equilibrium properties within a new hadron transport approach for heavy-ion collisions''~\cite{Weil:2016zrk}
    \item ``Forced canonical thermalization in a hadronic transport approach at high density''~\cite{Oliinychenko:2016vkg}
    \item ``Effective dynamical coupling of hydrodynamics and transport for heavy-ion collisions''~\cite{Oliinychenko:2017qct}
  \end{itemize}

%\section{Einf\"uhrung}
\selectlanguage{ngerman}

Die Dichte des Atomkerns ist f\"ur alle Kerne ann\"ahernd gleich und betr\"agt rund
$\rho_0 = 0.16$ fm$^{-3} = $ 2.7 $\cdot 10^{17}$ $\frac{\mathrm{kg}}{\mathrm{m}^3}$.
Unsere normale Raumtemperatur ist im Vergleich zu
nuklearen Energieniveaus so klein, dass man sie f\"ur Atomkerne als ann\"ahernd Null
betrachten kann. Es gibt allerdings Orte im Universum, an denen sich Kernmaterie in
einem extremen Zustand befindet. Zum Beispiel kann die Dichte beim Kollaps von Typ II
Supernovae bis zu $4\rho_0$ betragen, im Zentrum von Neutronensterne sogar bis zu
$9\rho_0$ und wenige Mikrosekunden nach dem Urknall war die
Materie nicht nur extrem dicht, die Temperatur war auch h\"oher als $10^{12}$ K.
In den 1970er Jahre wurde theoretisch vorhergesagt, dass bei solchen Temperaturen
und Dichten die gew\"ohnlichen Protonen und Neutronen nicht mehr existieren k\"onnen.
Ein neuer Zustand der Materie werde erzeugt, das sogenannte Quark-Gluon-Plasma
(QGP).

Das QGP kann man experimentell in hoch-energetischen Kollisionen von schweren Ionen
untersuchen, was manchmal ``Urknall im Labor'' genannt wird. Eine Reihe von
Experimenten an Ionenbeschleunigeranlagen sind der Erforschung der QGP-Physik und der
Materie bei extremen Dichten und Temperaturen gewidmet. Diese
Beschleuniger sind der RHIC (Relativistic Heavy Ion Collider) in Brookhaven bei New
York, der LHC (Large Hadron Collider) in Genf, sowie auch die zuk\"unftige
Beschleunigeranlagen FAIR (Facility for Antiproton and Ion Research) an der GSI
(Gesellschaft f\"ur Schwerionenforschung) in Darmstadt, NICA (Nuclotron-based Ion
Collider fAcility) in Dubna (Russland) und JPARK-HI in Japan. Dort werden verschiedene
Ionen auf ultrarelativistische Energien beschleunigt und zur Kollision gebracht. Die
Ergebnisse dieser Kollisionen erlauben es, die Eigenschaften der stark wechselwirkenden
Materie zu erforschen. Eine der wichtigsten Fragen der Schwerionenforschung heutzutage
ist, ob es einen Phasen\"ubergang zwischen hadronischer Materie und QGP gibt und falls
ja, bei welcher Temperatur und Dichte dieser erfolgt. Wenn es einen Phasen\"ubergang
gibt, stellt sich auch die Frage, welcher Ordnung er ist und wo der kritische Punkt
liegt. Durch Variation der Kollisionsenergie kann man verschiedene Punkte des
Phasendiagramms erreichen. Man erwartet, dass der kritische Punkt bei mittleren
Energien, $\Elab \simeq 20$--200~\emph{A}~GeV, beobachtet werden kann. Die
zuk\"unftigen Experimente, und auch das Beam Energy Scan Program am RHIC, arbeiten in
diesem Energiebereich, um den kritischen Punkt zu untersuchen. Das macht die
detaillierte Simulation von Schwerionenkollisionen bei mittleren Energien aktuell und
interessant.

Heutzutage gibt es zwei Arten von dynamischen mikroskopischen Modellen zur
Beschreibung von Schwerionenkollisionen: Relativistische Hydrodynamik und
Transportsimulationen. Modelle, die Hydrodynamik und Transporttheorie in ihrem
jeweiligen Anwendungsbereich verbinden, werden als Hybrid-Modelle bezeichnet.
Hydrodynamik beschreibt experimentelle Observablen besonders gut bei hohen 
Kollisionsenergien. Ein lokales thermodynamisches Gleichgewicht, welches notwendig
f\"ur die Anwendbarkeit relativistischer Hydrodynamik ist, wird bei hoher Energie
schnell erreicht und lange erhalten. Das System ist gro{\ss} und dicht genug, sodass
die mittlere freie Wegl\"ange viel kleiner als die Systemgr\"o{\ss}e ist und f\"ur eine
erhebliche Zeit erhalten bleibt. Im Gegensatz dazu ist hadronischer Transport bei
kleinen Energien anwendbar, kalibriert und relativ gut verstanden. Welche Methode kann
man f\"ur die mittlere Energie w\"ahlen? Sind die typische N\"aherungen und Vermutungen
der Hybrid-Modelle noch g\"ultig im mittleren Energiebereich? Diese N\"aherungen werden
in dieser Arbeit untersucht und beurteilt. Zus\"atzlich wird eine neue
Simulationsmethode vorgestellt, die es erlaubt, manche Annahmen wegzulassen.

Eine wichtige Annahme von Hybrid-Modellen ist eine schnelle lokale Thermalisierung
im gesamten Volumen der Reaktion. Bei hoher Energie ist die schnelle Ann\"aherung zum
thermischen Gleichgewicht gut erforscht und begr\"undet (obwohl die Diskussion
dar\"uber, welcher Thermalisierung-Mechanismus dominiert, noch sehr aktiv ist). Bei
mittleren Energien ist die Ann\"aherung zum Gleichgewicht weniger gut erforscht.
Allerdings ist die N\"ahe zum lokalen thermischen Gleichgewicht eine notwendige
Bedingung f\"ur die Anwendbarkeit von Hydrodynamik. Ist das Gleichgewicht bei mittleren
Energien \"uberhaupt erreicht? Wenn ja, wie schnell? Auf diese Fragen wird hier
mit Hilfe eines Transport-Modells UrQMD (Ultra-relativistic Quantum Molecular Dynamics)
sowie des sogenannten ``coarse-graining''-Verfahrens eingegangen. Hierf\"ur wird der
Energie-Impuls-Tensor $\Tmn$ f\"ur jeden Punkt des kartesischen Gitters bestimmt, das
sich \"uber das ganze System erstreckt. Mittels $\Tmn$ wird die Abweichung vom
Gleichgewicht bestimmt. Der wichtigste Beitrag zum Ungleichgewicht ist die Anisotropie
von $\Tmn$. Das Gleichgewicht wird nie im gesamten System erreicht, jedoch ist ab einer
bestimmten Zeit $t_{iso}(\Elab, b, \sigma)$ ein ausreichendes Volumen genug
isotropisiert, um Hydrodynamik anwenden zu k\"onnen. Dabei ist $\Elab$ die kinetische
Energie des Projektils pro Nukleon, $b$ der Sto{\ss}parameter, der die Zentralit\"at
der Kollision charakterisiert, und $\sigma$ ist ein Ausschmierung-Parameter des
coarse-graining-Verfahrens. Die gefundene Abh\"angigkeit (f\"ur eine Vielzahl der
Transport-Simulationen, ermittelt im coarse-graining-Verfahren) l\"asst sich
n\"aherungsweise mit der folgenden Formel beschreiben: $t_{iso} =
2R(\Elab/2m_N)^{-1/2} +\alpha \sigma$.

Hybrid-Modelle verwenden Hydrodynamik im Bereich hoher Dichten und Transport-
Simulationen im Bereich niedriger Dichten. Der \"Ubergang zwischen diesen zwei
Verfahren beruht auf bestimmten N\"aherungen und Annahmen. Man vermutet, dass dieser
\"Ubergang sehr schnell ist und auf einer Hyperfl\"ache erfolgt. Die hydrodynamischen
Gleichungen werden
im ganzen Vorw\"artslichtkegel gel\"ost und die Hyperfl\"ache wird
nur aus der Hydrodynamik aposteriori bestimmt, nicht dynamisch aus kombinierten
Hydrodynamik und Transporttheorie Gleichungen. Die Teilchen werden aus der
Hydrodynamik gem\"a{\ss} der Cooper-Frye-Formel produziert und anschlie{\ss}end im
Rahmen der Transporttheorie beschrieben. Sie k\"onnen nicht in den hydrodynamischen
Bereich zur\"uckkehren und r\"ucksto{\ss}en. Diese N\"aherungen f\"uhren zu negativen
Beitr\"agen der Cooper-Frye-Formel. Bei hohen Energien sind diese Beitr\"age
vernachl\"assigbar, aber bei mittleren Energien wurden sie nie systematisch untersucht.
In dieser Arbeit werden die negative Cooper-Frye-Beitr\"age in Gold-Gold Kollisionen bei
$\Elab = 5$--$160$ \emph{A} GeV f\"ur verschiedene Hadronen in verschiedenen
kinematischen Regionen mithilfe des ``coarse-grained'' Transport-Modells UrQMD bestimmt.
Diese Rechnung nimmt ein thermisches Gleichgewicht auf der Hyperfl\"ache
an. Die gr\"o{\ss}ten negative Beitr\"age liegen f\"ur Pionen bei mittlerer Rapidit\"at
vor und machen nicht mehr als 15\% aus. In Transport-Modellen kann man auch explizit
die Teilchen z\"ahlen, welche die Hyperfl\"ache von au{\ss}en nach innen \"uberqueren
--- das entspricht  den negativen Cooper-Frye-Beitr\"agen ohne Annahme eines
Gleichgewichts. In dieser Arbeit wird gezeigt, dass diese negativen Beitr\"age im
Nichtgleichgewicht erheblich kleiner sind als im Gleichgewicht.

Die negativen Cooper-Frye-Beitr\"age vermeidet man in einem neuen Modell, das in dieser
Arbeit konstruiert wird. In gew\"ohnlichen Transport-Modellen im Bereich hoher Dichten
wird eine Thermalisierung k\"unstlich erzwungen, was intensiven Mehr-Teilchen-
Kollisionen oder der Bildung des Quark-Gluon-Plasmas entspricht. Dieses Verfahren wurde
mit dem Transport-Modell SMASH implementiert und getestet. Drei Algorithmen der
Thermalisierung wurden verglichen und der neuentwickelte ``biased Becattini-Ferroni''
Algorithmus erwies sich als ausreichend zuverl\"assig und effizient. In einem
kontrollierten Szenario einer expandierenden Kugel wurde gezeigt, dass SMASH mit
erzwungener Thermalisierung die Expansionsgeschwindigkeit und die Energiedichte zwischen
Hydrodynamik und Transporttheorie aufweist. Bei der Simulation von Schwerionenkollisionen mit
diesem Modell wurden folgende Beobachtungen gemacht: im Vergleich zum Transport wird
mehr Seltsamkeit erzeugt, der mittlere transversale Impuls wird aufgrund der
Druckisotropisierung erh\"oht und Bereiche hoher Dichten leben l\"anger. Alle diese
Merkmale sind den Hybrid-Modellen \"ahnlich, aber ohne die oben genannten Nachteile.
Insgesamt f\"uhrt die erzwungene Thermalisierung zu den erwarteten Ergebnissen.

Im Rahmen dieser Doktorarbeit wurde auch ein erheblicher Beitrag zur Enwicklung des
SMASH-Modells selbst geleistet, inbesonders was Nukleon-Nukleon-Potentiale,
Fermi-Bewegung, Pauli-Blockierung, Thermodynamik und Detaillierte Balance betrifft. 
Das relativistische Transport-Modell SMASH ist f\"ur Simulationen der Kollisionen von
Ionen, Protonen oder anderen Hadronen geeignet. Ein gesamter Kollisionsprozess wird als
eine Sequenz von elementaren $2 \to 2$ Kollisionen, Zerf\"allen, Resonanz-Bildungen
und Teilchen-Propagation simuliert. Die Potentiale beeinflussen die Trajektorien der
Teilchen w\"ahrend der Propagation und sind f\"ur Schwerionenkollisionen niedriger
Energien (z.B. f\"ur Simulationen des FOPI Experiments an der GSI in Darmstadt)
besonders wichtig. Bei solchen Energien ist es auch wichtig die Quanteneffekte f\"ur
Fermionen, die Pauli-Blockierung und Fermi-Bewegung in einem Atomkern zu
ber\"ucksichtigen. Zus\"atzlich zu diesen Effekten wurde das oben genannte ``coarse-
graining''-Verfahren in SMASH implementiert. Das hat die oben beschriebenen
Simulationen mit erzwungener Thermalisierung erm\"oglicht.

Zusammengefasst lauten die wichtigsten Resultate dieser Arbeit:
\begin{itemize}
  \item Entwicklung eines neuen Modells mit erzwungener Thermalisierung im Bereich
        hoher Dichten. Diese Thermalisierung entspricht intensiven Mehr-Teilchen-
        Kollisionen oder der Bildung des Quark-Gluon-Plasmas.
  \item Systematische Beurteilung der negativen Cooper-Frye-Beitr\"age in
        Schwerionenkollisionen im Energiebereich $\Elab = 5$--$160$ \emph{A} GeV.
        Die negativen Cooper-Frye Beitr\"age limitieren die Pr\"azision der 
        Hybrid-Modelle in diesem Bereich.
  \item Analyse der lokalen Thermalisierungsgrade in Schwerionenkollisionen im
        Energiebereich $\Elab = 5$--$160$ \emph{A} GeV.
  \item Erheblicher Beitrag zu der Entwicklung des Transport-Modells SMASH.
\end{itemize}

% Das SMASH-Modell wird in der Arbeit ausf\"uhrlich beschrieben.

\selectlanguage{english}

\end{abstract}


%% Declaration
%\begin{declaration}
%  This dissertation is the result of my own work, except where explicit
%  reference is made to the work of others, and has not been submitted
%  for another qualification to this or any other university. This
%  dissertation does not exceed the word limit for the respective Degree
%  Committee.
%  \vspace*{1cm}
%  \begin{flushright}
%    Dmytro Oliinychenko
%  \end{flushright}
%\end{declaration}


%% ToC
\tableofcontents


%%
% \frontquote{<quote>}{<author>}%
%% I don't want a page number on the following blank page either.
\thispagestyle{empty}
