\chapter{Summary}
\label{chap:summary}

This thesis explores the limitations of traditional hybrid (hydrodynamics +
transport) approaches for simulations of heavy ion collisions at beam energies
$\Elab = $ 5--160 \emph{A} GeV and suggests a new forced thermalization
approach, where some of these limitations are relaxed. After a brief
introduction, which motivates heavy ion collisions as an experimental approach
to study the phase diagram of a strongly-interacting matter, a small overview
of experimental facilities is given. It demonstrates a continuing significant
experimental interest to the intermediate energy range studied in this thesis.

Relativistic hydrodynamics and transport are introduced, mentioning that
hydrodynamics is calibrated and describes experiments very well at highest RHIC
and LHC energies, while hadronic transport is particularly good at low
energies. A reliable approach at intermediate energies is an important goal:
one can either extrapolate the hydrodynamical and hybrid approaches from higher
energies or transport approaches from lower energies. In this thesis
the approximations made by traditional hybrid approaches are tested and it is argued
that they become challenging at intermediate energies.

Switching from transport to hydrodynamics requires certain degree of equilibration.
Here the degree of local equilibration was quantified for Au+Au collisions at $\Elab =
$ 5--160 \emph{A} GeV using coarse-grained transport approach. It was found that the
most important contribution to deviation from equilibrium comes from pressure
anisotropy. Other effects could be diminished if many events from transport code were
used for fluidization. However, if one constructs an initial state for hydrodynamics
from single transport events, not only pressure anisotropy is large, but also
off-diagonal components of energy-momentum tensor deviate significantly from
equilibrium. In general, it is not obvious whether it is always the case that
the degree of isotropization necessary
to switch to hydrodynamics is reached in transport approach at any time. Here it was
shown that it is indeed reached in a significant volume, although somewhat later than
hybrid models typically perform fluidization. The isotropization time depends on the
smearing parameter $\sigma$ as $t_{iso} = t_{geom} + \alpha \sigma$, where $t_{geom}$ is
the time of geometrical overlap and $\alpha$ is a proportionality coefficient. It
partially justifies the typical convention of hybrid approaches to perform fluidization
at $t_{geom}$.

In the hybrid approaches hydrodynamical equations are solved in the whole forward light
cone including those regions, where hydrodynamics is not applicable. The switching
hypersurface is chosen aposteriori from the hydrodynamical evolution, but not
from coupled hydrodynamical and transport equations. The particles from
transport cannot cause any feedback to hydrodynamics. These approximations are
manifested in the negative contributions to Cooper-Frye formula. Here it is shown that
the negative Cooper-Frye contributions at midrapidity become significant for
intermediate energies, which motivates new approaches, that avoid the negative
Cooper-Frye contributions problem.

Such an approach was suggested as a main result of this thesis. In this novel approach
conventional transport simulation is performed, but at the regions of high energy
density, where traditional hybrid model would switch to hydrodynamics, a forced
canonical thermalization is applied. The approach was tested in an artificial setup
and in low energy heavy ion collisions and has shown reasonable results. The task
still remains to apply it to intermediate energies confronting its results to
experimental data.



