%% Acknowledgements
\begin{acknowledgements}

  First of all I would like to thank Hannah Petersen for the brilliant
supervision, for being always kind, optimistic and full of ideas, for
motivating me, for interesting discussions and for the help in all sorts of
weird situations one can get into, while doing PhD abroad.

This work has inherited many ideas developed by Kyrill Bugaev and by Pasi
Huovinen.  I would like to thank them for that, as well as for insightful
discussions and critical questions. I would also like to thank Pasi for
providing the Cornelius code for hypersurface finding. The thesis is influenced
by the excellent lectures in heavy ion physics by Elena Bratkovskaya and Igor
Mishustin that I attended in the Institute of Theoretical Physics.  Discussions
during common lunches with Harri Niemi, Etele Molnar, LongGang Pang, Pasi
Huovinen, Daisuke Satow, Shi Pu and many others were illuminating and certainly
contributed to my understanding of heavy ion collisions.

Helmholtz Graduate School has supported my PhD studies, providing
the travel money and excellent lectures. I would like to thank Henner B\"usching
and Sascha Vogel, who organized these lectures and motivated us to visit them.
Due to the requirement of Helmholtz Graduate School to have a PhD committee I
had a chance to discuss my work with Christian Sturm - he helped me not to lose
the big picture behind technical details.

I would like to acknowledge funding by a Helmholtz Young Investigator
Group VH-NG-822 from the Helmholtz Association and GSI.

For two years my PhD studies were supported by the stipend from Deutsche Telekom
Stiftung. At the get-togethers of Deutsche Telekom Stiftung stipendiats I was
introduced to a world of socially active students from natural sciences that changed my
mentality. I would like to thank all fellow stipendiats for that, especially Arzu
Yilmas, Niklas M\"uller and Isabel M\"uller.  The help and support of Christiane
Frense-Heck, who is at the heart of organizational activities, cannot be overestimated.

It was exciting to be a part of the SMASH developers group. I want to thank all
its members for a great collaboration. Besides co-developing SMASH, many have
contributed to this thesis in different ways. Thanks to Matthias Kretz I have
learned a lot about modern C++ programming.  Vinzent Steinberg and LongGang Pang
were always willing to discuss scientific and technical issues, their support
helped me to avoid and correct many errors. I would like to thank Harri Niemi for
his critical comments on the thesis, Anna Sch\"afer, Alba Ontoso and Sangwook Ryu for
proof-reading different parts of the the thesis and Juan Torres-Rincon for correcting
typos in formulas. Markus Mayer and Vinzent Steinberg have corrected my German in the
summary (Zusammenfassung).

Significant part of the computations for this thesis was performed on the
LOEWE-CSC computing cluster, which can be praised for its reliability and
impressive computational power.

Finally, I would like to thank my parents, who always cared and worried that
the author of this thesis is alive, healthy, fit, and has a good mood for some
scientific research.


\end{acknowledgements}

%% You're recommended to use the eprint-aware biblio styles which
%% can be obtained from e.g. www.arxiv.org. The file mythesis.bib
%% is derived from the source using the SPIRES Bibtex service.
\bibliographystyle{IEEEtran}
\bibliography{inspire,non_inspire}

%% I prefer to put these tables here rather than making the
%% front matter seemingly interminable. No-one cares, anyway!
%\listoffigures
%\listoftables

%% If you have time and interest to generate a (decent) index,
%% then you've clearly spent more time on the write-up than the
%% research ;-)
%\printindex

\begin{lebenslauf}

\begin{tabular}{ll}
\textbf{Name:}           & Dmytro Oliinychenko \\
\textbf{Adresse:}        & Augustenburgstr. 6, 60488 Frankfurt am Main \\
\textbf{Geburtsdatum:}   & 8. November 1990 \\
\textbf{Geburtsort:}     & Dnepropetrovsk, Ukraine \\
\textbf{Nationalit\"at:} & ukrainisch \\
\textbf{Familienstand:}  & ledig \\
\end{tabular}

\vspace{0.5cm}
{\Large \textbf{Ausbildung}} \\

\begin{tabular}{m{4cm}l}
9/1997 - 7/2004  & Rivne State Classical Gymnasium \\
9/2004 - 7/2007  & Lyceum at Kiev National University \\
9/2007 - 7/2013  & Moscow Institute for Physics and Technology \\
\end{tabular}

\begin{itemize}[leftmargin=2.5cm,label=$-$]
  \item B. S. in Physik, Summa Cum Laude \\
        {\small \textit{Undergraduate Thesis: ''Description of particle multiplicities
         in heavy ion collisions data using thermal model with hard-core potential
         including Lorentz contraction''}}
  \item M. S. in Physik, Summa Cum Laude \\
        {\small \textit{Graduate Thesis: ''Investigation of Multiplicities in Heavy Ion
        Collisions''}}
\end{itemize}

\begin{tabular}{m{4cm}l}
seit 10/2013 & Doktorand in Physik an der J.W.-Goethe Universit\"at Frankfurt
\end{tabular}

\vspace{0.5cm}
{\Large \textbf{Stipendien und Auszeichnungen}} \\

\begin{tabular}{m{3cm}m{13cm}}
2006  & Zweiter Preis in der Ukrainischen Physik Olympiade (Nationalniveau) \\
2007  & Stipendium des Presidenten der Ukraine f\"ur Gewinner der Schulolympiaden \\
2010, 2012, 2013  & Gewinner/''Best reporter'' des Internationalen Physiker-Turniers \\ % sieh iptnet.info
2008 - 2013 & Stipendium des Dekanats f\"ur ausgezeichnete Pr\"ufungsleistung \\
2012 - 2013 & Stipendium vom D. Zimin ''Dynasty''-Stiftung f\"ur junge Forscher \\
2013        & M. M. Bogoliubov Stipendium f\"ur junge Forscher vom JINR, Dubna \\
2015        & Giersch Excellence Award \\
2015 - 2016 & Stipendium der Deutsche Telekom Stiftung  \\
\end{tabular}

\newpage

{\Large \textbf{Ver\"offentlichungen}} \\

\textbf{In referierten Zeitschriften}

\begin{enumerate}
    \item A.~I.~Ivanytsky \textit{et al.},
          \emph{Physical properties of Polyakov loop geometrical clusters in SU(2)
                gluodynamics},
          Nucl.\ Phys.\ A \textbf{ 960}, 90-113 (2017).
    \item D.~Oliinychenko and H.~Petersen,
          \emph{Forced canonical thermalization in a hadronic transport approach at high
                density},
          J.\ Phys.\ G \textbf{ 44}, no. 3, 034001 (2017).            
    \item K.~A.~Bugaev \textit{et al.},
          \emph{Separate chemical freeze-outs of strange and non-strange hadrons and problem of
                residual chemical
                non-equilibrium of strangeness in relativistic heavy ion collisions},
          Ukr.\ J.\ Phys.\  \textbf{ 61}, 659 (2016).
    \item J.~Weil \textit{et al.},
          \emph{Particle production and equilibrium properties within a new hadron transport
                approach for heavy-ion collisions},
          Phys.\ Rev.\ C \textbf{ 94}, no. 5, 054905 (2016).
    \item  K.~A.~Bugaev, V.~V.~Sagun, A.~I.~Ivanytskyi, D.~R.~Oliinychenko, E.~M.~Ilgenfritz,
           E.~G.~Nikonov, A.~V.~Taranenko and G.~M.~Zinovjev,
           \emph{New Signals of Quark-Gluon-Hadron Mixed Phase Formation},
           Eur.\ Phys.\ J.\ A \textbf{ 52}, no. 8, 227 (2016).
    \item H.~Petersen, D.~Oliinychenko, J.~Steinheimer and M.~Bleicher,
          \emph{Influence of kinematic cuts on the net charge distribution},
          Quark Matter 2016 proceedings, Nucl.\ Phys.\ A \textbf{ 956}, 336 (2016).
    \item D.~Oliinychenko and H.~Petersen,
          \emph{Deviations of the Energy-Momentum Tensor from Equilibrium in the Initial State
                for Hydrodynamics from Transport Approaches},
          Phys.\ Rev.\ C \textbf{ 93}, no. 3, 034905 (2016).
    \item K.~A.~Bugaev, A.~I.~Ivanytskyi, D.~R.~Oliinychenko, V.~V.~Sagun, I.~N.~Mishustin,
          D.~H.~Rischke, L.~M.~Satarov and G.~M.~Zinovjev,
          \emph{Thermodynamically Anomalous Regions and Possible New Signals of Mixed Phase
                Formation},
          Eur.\ Phys.\ J.\ A \textbf{ 52}, no. 6, 175 (2016).
    \item D. Oliinychenko, P. Huovinen, H. Petersen,
          \emph{Systematic Investigation of Negative Cooper-Frye Contributions in Heavy Ion
                Collisions Using Coarse-grained Molecular Dynamics},
          Phys. Rev. C \textbf{ 91} (2015) 2, 024906.
    \item K. A. Bugaev, A. I. Ivanytskyi, D. R. Oliinychenko et al.,
          \emph{Thermodynamically Anomalous Regions as a Mixed Phase Signal},
          Phys. Part. Nucl. Lett. \textbf{ 12} (2015) 2, 238-245.
    \item K. A. Bugaev, A. I. Ivanytskyi, D. R. Oliinychenko et al.,
          \emph{Non-smooth Chemical Freeze-out and Apparent Width of Wide Resonances and Quark
                Gluon Bags in a Thermal Environment},
          Ukr. J. Phys. \textbf{ 60} (2015) 3, 181-200.
    \item V. V. Sagun, D. R. Oliinychenko, K. A. Bugaev et al.,
          \emph{Strangeness enhancement at the hadronic chemical freeze-out},
          Ukr. J. Phys. \textbf{ 59} (2014) 1043-1050.
    \item D. R. Oliinychenko, V. V. Sagun, A. I. Ivanytskyi, K. A. Bugaev,
          \emph{Separate chemical freeze-out of strange particles with conservation laws},
          Ukr. J. Phys. \textbf{ 59} (2014) 1051-1059.
    \item K. A. Bugaev, D. R. Oliinychenko, et al.,
    	  \emph{Chemical freeze-out of strange particles and possible root of strangeness
                suppression},
          Europhys. Lett., \textbf{ 104} (2013) 22002.
    \item K. A. Bugaev, D. R. Oliinychenko, A. S. Sorin,
          \emph{A Scientific Analysis of the Preprint arXiv:1301.1828v1  [nucl-th]},
          Ukr. J. Phys. \textbf{ 58} (2013) 10, 939-943.
    \item K. A. Bugaev, A. I. Ivanytskyi, V. V. Sagun and D. R. Oliinychenko,
          \emph{ Is  bimodality a sufficient condition for a first order phase transition
                existence?},
          Phys.  Part. Nucl. Lett. \textbf{ 10} (2013), 6, pp. 832-851.
    \item K. A. Bugaev, D. R. Oliinychenko, A. S. Sorin, G. M. Zinovjev,
          \emph{Simple Solution to the Strangeness Horn Description Puzzle},
	      Eur. Phys. J. A \textbf{ 49} (2013), 30--1-8.
    \item D.R. Oliinychenko, K.A. Bugaev, A.S. Sorin,
          \emph{Investigation of hadron multiplicities and hadron yield ratios in heavy ion
                collisions},
	      Ukr. J. Phys. \textbf{ 58} (2013), No. 3, 211-227.
\end{enumerate}

\textbf{Konferenz Proceedings}

\begin{enumerate}
    \item D.~Oliinychenko and H.~Petersen,
          \emph{Effective dynamical coupling of hydrodynamics and transport for heavy-ion
           collisions},
          J.\ Phys.\ Conf.\ Ser.\  \textbf{ 832}, no. 1, 012052 (2017).
    \item A.~Ivanytskyi \textit{et al.},
          \emph{Geometrical clusterization of Polyakjov loops in SU(2) lattice gluodynamics},
          J.\ Phys.\ Conf.\ Ser.\  \textbf{ 798}, no. 1, 012065 (2017)
    \item V.~V.~Sagun, K.~A.~Bugaiev, A.~I.~Ivanytskyi, D.~R.~Oliinychenko and I.~N.~Mishustin,
          \emph{Effects of Induced Surface Tension in Nuclear and Hadron Matter},
          EPJ Web Conf.\ \textbf{ 137}, 09007 (2017)
    \item D.~Oliinychenko, H.~Petersen,
          \emph{Effective dynamical coupling of hydrodynamics and transport for heavy-ion
                collisions},
          Hot Quarks 2016, to be published in Journal of Physics: Conference Series (JPCS).
    \item V.~V.~Sagun, K.~A.~Bugaiev, A.~I.~Ivanytskyi, D.~R.~Oliinychenko and I.~N.~Mishustin,
          \emph{Effects of Induced Surface Tension in Nuclear and Hadron Matter},
          XII Quark Confinement and Hadron Spectrum Conference (CONF12), arXiv:1611.07071
          [nucl-th].
    \item V.~V.~Sagun, K.~A.~Bugaev, A.~I.~Ivanytskyi and D.~R.~Oliinychenko,
          \emph{New irregularities at chemical freeze-out of hadrons},
          2015 International Young Scientists Forum on Applied Physics (YSF),
          doi:10.1109/YSF.2015.7333191
    \item K.~A.~Bugaev, A.~I.~Ivanytskyi, V.~V.~Sagun, G.~M.~Zinovjev, D.~R.~Oliinychenko,
          V.~S.~Trubnikov and E.~G.~Nikonov,
          \emph{A possible evidence of observation of two mixed phases in nuclear collisions},
          EPJ Web Conf.\  \textbf{ 126}, 03003 (2016).
    \item D. Oliinychenko, P. Huovinen, H. Petersen,
          \emph{Cooper-Frye Negative Contributions in a Coarse-Grained Transport Approach},
          FAIRNESS 2014, J.\ Phys.\ Conf.\ Ser.\  \textbf{ 599}, no. 1, 012017 (2015).    
    \item V.~V.~Sagun, A.~I.~Ivanytskyi, D.~R.~Oliinychenko and K.~A.~Bugaev,
          \emph{Bimodality as a signal of the nuclear liquid-gas phase transition},
          arXiv:1401.2881 [nucl-th].
    \item V.~V.~Sagun, A.~I.~Ivanytskyi, K.~A.~Bugaev and D.~R.~Oliinychenko,
          \emph{Bimodality Phenomenon in Finite and Infinite Systems Within an Exactly Solvable
          Statistical Model}
          Helmholtz International Summer School "Physics of Heavy Quarks and Hadrons", Dubna,
          Russia, July 15-28, 2013;
          arXiv:1311.7042 [nucl-th].
    \item V.~V.~Sagun, A.~I.~Ivanytskyi, D.~R.~Oliinychenko and K.~A.~Bugaev,
	      \emph{Can bimodality exist without phase transition?},
	      XI International Scientific Conference of Students and Young
	      Scientists "Shevchenkivska Vesna 2013", Kyiv, March 18-22, 2013;
	      arXiv:1304.5997 [nucl-th].
    \item K. A. Bugaev, D. R. Oliinychenko, A. S. Sorin, E.G. Nikonov, G. M. Zinovjev,
          \emph{Adiabatic chemical freeze-out and wide resonance modification in a thermal
          medium},
          {PoS Baldin ISHEPP XXI} (2012) \textbf{ 017}, 1-14; arXiv:1212.0132 [hep-ph].
\end{enumerate}

\vspace{0.5 cm}

{\Large \textbf{Lehre}} \\

\vspace{0.5 cm}

\begin{tabular}{m{3cm}m{13cm}}
WS 2016  & Tutorium, Theoretische Physik I \\
WS 2017  & Tutorium, Theoretische Physik V (Statistische Physik) \\
\end{tabular}

\end{lebenslauf}


