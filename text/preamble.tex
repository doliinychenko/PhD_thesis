\usepackage{xspace}
\usepackage{tikz}
\usepackage{morefloats,subfig,afterpage}
\usepackage{mathrsfs} % script font
\usepackage{verbatim}
\usepackage{amssymb}

%% Using Babel allows other languages to be used and mixed-in easily
\usepackage[ngerman,english]{babel}
\selectlanguage{english}

%% Citation system tweaks
\usepackage{cite}
% \let\@OldCite\cite
% \renewcommand{\cite}[1]{\mbox{\!\!\!\@OldCite{#1}}}

%% Maths
% TODO: rework or eliminate maybemath
\usepackage{abmath}
\DeclareRobustCommand{\parenths}[1]{\left({#1}\right)\xspace}
\DeclareRobustCommand{\braces}[1]{\left\{{#1}\right\}\xspace}
\DeclareRobustCommand{\angles}[1]{\left\langle{#1}\right\rangle\xspace}
\DeclareRobustCommand{\sqbracs}[1]{\left[{#1}\right]\xspace}
\DeclareRobustCommand{\mods}[1]{\left\lvert{#1}\right\rvert\xspace}
\DeclareRobustCommand{\modsq}[1]{\mods{#1}^2\xspace}
\DeclareRobustCommand{\dblmods}[1]{\left\lVert{#1}\right\rVert\xspace}
\DeclareRobustCommand{\expOf}[1]{\exp{\!\parenths{#1}}\xspace}
\DeclareRobustCommand{\eexp}[1]{e^{#1}\xspace}
\DeclareRobustCommand{\plusquad}{\oplus\xspace}
\DeclareRobustCommand{\logOf}[1]{\log\!\parenths{#1}\xspace}
\DeclareRobustCommand{\lnOf}[1]{\ln\!\parenths{#1}\xspace}
\DeclareRobustCommand{\ofOrder}[1]{\mathcal{O}\parenths{#1}\xspace}
\DeclareRobustCommand{\SOgroup}[1]{\mathrm{SO}\parenths{#1}\xspace}
\DeclareRobustCommand{\SUgroup}[1]{\mathrm{SU}\parenths{#1}\xspace}
\DeclareRobustCommand{\Ugroup}[1]{\mathrm{U}\parenths{#1}\xspace}
\DeclareRobustCommand{\I}[1]{\mathrm{i}\xspace}
\DeclareRobustCommand{\colvector}[1]{\begin{pmatrix}#1\end{pmatrix}\xspace}
\DeclareRobustCommand{\Re}[1]{\operatorname{Re}(#1)}
\DeclareRobustCommand{\Im}[1]{\operatorname{Im}(#1)}

%% High-energy physics stuff
\usepackage{abhep}
\usepackage{hepnames}
\usepackage{hepunits}

\DeclareRobustCommand{\Elab}{E_\mathrm{lab}}
\DeclareRobustCommand{\Tmn}{T^{\mu\nu}}
\DeclareRobustCommand{\jmu}{j^{\mu}}
\DeclareRobustCommand{\U}{\mathrm{U}}
\DeclareRobustCommand{\Ntest}{N_{\textrm test}}

\DeclareRobustCommand{\arXivCode}[1]{arXiv:#1}
\DeclareRobustCommand{\CP}{\ensuremath{\mathcal{CP}}\xspace}
\DeclareRobustCommand{\CPviolation}{\CP-violation\xspace}
\DeclareRobustCommand{\CPv}{\CPviolation}
\DeclareRobustCommand{\LHCb}{LHCb\xspace}
\DeclareRobustCommand{\LHC}{LHC\xspace}
\DeclareRobustCommand{\LEP}{LEP\xspace}
\DeclareRobustCommand{\CERN}{CERN\xspace}
\DeclareRobustCommand{\bphysics}{\Pbottom-physics\xspace}
\DeclareRobustCommand{\bhadron}{\Pbottom-hadron\xspace}
\DeclareRobustCommand{\Bmeson}{\PB-meson\xspace}
\DeclareRobustCommand{\bbaryon}{\Pbottom-baryon\xspace}
\DeclareRobustCommand{\Bdecay}{\PB-decay\xspace}
\DeclareRobustCommand{\bdecay}{\Pbottom-decay\xspace}
\DeclareRobustCommand{\BToKPi}{\HepProcess{ \PB \to \PK \Ppi }\xspace}
\DeclareRobustCommand{\BToPiPi}{\HepProcess{ \PB \to \Ppi \Ppi }\xspace}
\DeclareRobustCommand{\BToKK}{\HepProcess{ \PB \to \PK \PK }\xspace}
\DeclareRobustCommand{\BToRhoPi}{\HepProcess{ \PB \to \Prho \Ppi }\xspace}
\DeclareRobustCommand{\BToRhoRho}{\HepProcess{ \PB \to \Prho \Prho }\xspace}
\DeclareRobustCommand{\X}{\thesismath{X}\xspace}
\DeclareRobustCommand{\Xbar}{\thesismath{\overline{X}}\xspace}
\DeclareRobustCommand{\Xzero}{\HepGenParticle{X}{}{0}\xspace}
\DeclareRobustCommand{\Xzerobar}{\HepGenAntiParticle{X}{}{0}\xspace}
\DeclareRobustCommand{\epluseminus}{\Ppositron\!\Pelectron\xspace}
\DeclareRobustCommand{\protonproton}{\Pproton\APantiproton\xspace}


\usepackage{etoolbox}
\newcommand{\zerodisplayskips}{%
  \setlength{\abovedisplayskip}{-5pt}%
  \setlength{\belowdisplayskip}{5pt}%
  \setlength{\abovedisplayshortskip}{-5pt}%
  \setlength{\belowdisplayshortskip}{5pt}}
\appto{\normalsize}{\zerodisplayskips}
\appto{\small}{\zerodisplayskips}
\appto{\footnotesize}{\zerodisplayskips}

\usepackage{enumitem}
\setlist[1]{itemsep=0pt}

\usepackage{titlesec}
\titlespacing\section{0pt}{12pt plus 4pt minus 2pt}{0pt plus 2pt minus 2pt}
\titlespacing\subsection{0pt}{12pt plus 4pt minus 2pt}{0pt plus 2pt minus 2pt}
\titlespacing\subsubsection{0pt}{12pt plus 4pt minus 2pt}{0pt plus 2pt minus 2pt}

\usepackage{array}
\usepackage{pbox}

